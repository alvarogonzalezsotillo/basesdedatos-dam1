\input{../common/plantilla-ejercicio.tex}
\usepackage{eurosym}





\renewcommand{\hmwkTitle}{Diseño E/R, paso a SQL e inserción de datos}


\usepackage{blindtext}

\begin{document}

% \maketitle

% ----------------------------------------------------------------------------------------
%	TABLE OF CONTENTS
% ----------------------------------------------------------------------------------------

% \setcounter{tocdepth}{1} % Uncomment this line if you don't want subsections listed in the ToC

\primerapagina

\setlength{\parindent}{2em}
\setlength{\parskip}{1em}

\section{Objetivo de la práctica}
Se pretende que el alumno sea capaz de pasar desde las necesidades de datos de una empresa hasta las órdenes SQL necesarias para implementar dicha necesidad en Oracle

\section{Modelo de datos}
Una compañía de alquiler de coches desea informatizar su gestión.
Su flota de coches se divide en tres categorías: Functional, Advance y
Prestige. Cada categoría puede tener diferentes modelos de coche
con diferentes acabados. Un modelo concreto puede pertenecer a
varias categorías, si el acabado es distinto. El precio por día del
alquiler depende de la categoría asignada al coche.

Se desea saber los siguientes datos en el acabado de los coches:
\begin{itemize}
\item aire acondicionado, climatizador, navegador, bluetooth, mp3,
  cambio automático, techo solar
\end{itemize}

De cada coche de la flota se desea saber:
\begin{itemize}
\item Marca
\item Modelo
\item Categoría
\item Acabado
\item Color de carrocería
\item Matrícula y número de bastidor
\end{itemize}

De cada alquiler de coche se desea saber:
\begin{itemize}
\item Datos del cliente (nombre, apellidos, DNI/NIE/pasaporte, edad,
  género)
\item  Delegación donde se recoge el coche, y delegación donde se
  dejará el coche
\item  Número de Km inicial y final
\item  Fecha de inicio y fin del alquiler
\item  Descuento comercial aplicado
\end{itemize}


\begin{homeworkProblem}[: Realizar el diagrama \textbf{ER} con Oracle \textbf{SQLDeveloper} (2 puntos)]
  En el diagrama deben ser visibles (al menos):
  \begin{itemize}
  \item Los nombres de las entidades
  \item Los atributos de las entidades
  \item Las relaciones entre las entidades
  \end{itemize}

\end{homeworkProblem}

\begin{homeworkProblem}[: Realizar el diagrama \textbf{relacional} con Oracle \textbf{SQLDeveloper} (1 punto)]
  Este paso puede conseguirse:
  \begin{itemize}
  \item De forma automática tras el ejercicio anterior, pidiendo a \textbf{Modeler} que convierta el diagrama \textbf{ER} en uno relacional.
  \item De forma manual
  \end{itemize}

  En el diagrama deben ser visibles (al menos):
  \begin{itemize}
  \item Los nombres de las tablas
  \item Los campos de las tablas, su tipo y si son claves primarias
  \item Las relaciones entre las tablas (conexiones debidas a claves extranjeras)  
  \end{itemize}
  
\end{homeworkProblem}


\begin{homeworkProblem}[: Crear las tablas de Oracle utilizando \textbf{SQL} (1 punto)]
  Este paso puede conseguirse:
  \begin{itemize}
  \item De forma automática tras el ejercicio anterior, pidiendo a \textbf{Modeler} que convierta el diagrama \textbf{relacinal} en las sentencias \textbf{DDL} (\textit{Data Definition Language}) de \textbf{SQL}
  \item De forma manual
  \end{itemize}

  Se entregará un fichero de texto de extensión \texttt{.SQL} con las órdenes \textbf{SQL} de creación de tablas, sus claves primarias, extranjeras, valores por defecto y restricciones.

\end{homeworkProblem}



\begin{homeworkProblem}[: Insertar datos en las tablas (2 puntos)]
  Se insertarán los siguientes datos en las tablas. Los campos no especificados se
  rellenarán a gusto del alumno:

  \begin{enumerate}
  \item  \texttt{Pedro Martínez Martínez} alquiló un Mercedes Clase B verde con
    Matrícula \texttt{1234ABC} el \texttt{1-1-12} al \texttt{1-2-13}. Tenía cambio automático, y
    realizó 200 km.
  \item  \texttt{Pedro Martínez Martínez} alquiló un Renault Fluenze con Matrícula
    \texttt{1111ABC} el \texttt{1-3-12} al \texttt{1-4-13}. Tenía navegador y bluetooth, y realizó
    100 km.
  \item  Los precios por día de las categorías Functional, Advance y Prestige
    son de 200\euro, 120\euro y 95\euro, respectivamente

  \item  Los coches con cambio automático o navegador se están clasificando
    como categoría \texttt{Prestige}. Los que tienen bluetooth, en la categoría
    \texttt{Advance}.
  \item  \texttt{Juan Pérez Pérez} alquiló un Renault Kangoo con Matrícula \texttt{4321ABC} el
    \texttt{1-3-12} al \texttt{1-4-13}. Tenía bluetooth, y se le aplicó un descuento del \texttt{5\%}.
    Recogió el coche en Madrid, y lo devolvió en Sevilla.
  \item  \texttt{Manolo Bombo Bombo} alquiló un Renault Modus el \texttt{1-3-13}, y aún no lo
    ha devuelto. Tenía bluetooth, y lo recogió en Barcelona.
  \end{enumerate}
\end{homeworkProblem}


\begin{homeworkProblem}[: Modificar las tablas (2 puntos)]

  Tras la puesta en marcha del sistema, la dirección de la empresa necesita modificar las tablas ya existentes para almacenar nueva información:
  
  \begin{itemize}
  \item La empresa se ha dado cuenta que necesita realizar revisiones
    periódicas a los coches. Necesita saber, por cada coche:
    \begin{itemize}
    \item Tipo de revisión (ITV, mantenimiento)
    \item Nº de kilómetros o fecha máxima para esa revisión.
    \item Nº de kilómetros o fecha en la que se pasaron las anteriores revisiones.
    \end{itemize}


  \item  Algunos coches se han repintado de otros colores. Se necesita saber qué colores ha tenido un coche, en qué fechas.
  \item  Se desea añadir un acabado para asientos calefactados.
  \end{itemize}
\end{homeworkProblem}

\begin{homeworkProblem}[: Inserción de nuevos datos (2 puntos)]

  \begin{itemize}
  \item El coche con matrícula \texttt{1234ABC} se compró \texttt{rojo} el \texttt{1-1-11}. Se repintó a \texttt{verde} el \texttt{1-6-11}.
  \item  Los Renault deben realizar un mantenimiento cada 15000 km.
    
  \item \texttt{1111ABC} realizó su primera revisión el \texttt{1-3-12}.
  \item \texttt{4321ABC} se compró el \texttt{1-2-12}, y lleva dos revisiones, el \texttt{1-4-12} y el \texttt{1-10-12}.

  \item Los Mercedes realizan un mantenimiento cada 20000 km. 
  \item \texttt{1234ABC}   se compró el \texttt{1-1-12}, y se revisó por primera vez el \texttt{1-4-12}.
  \item Todos los coches se revisan a los 4 años en la ITV.
  \item El coche con matrícula \texttt{1111ABC} tiene asientos calefactados.
  \end{itemize}
\end{homeworkProblem}

\section{Instrucciones de entrega}
\begin{itemize}
\item El ejercicio se realizará y entregará de manera individual.
  \begin{itemize}
  \item Solo se admiten trabajos en pareja, si en clase es necesario compartir ordenador.
  \end{itemize}
\item Entrega tu trabajo en un fichero \texttt{ZIP} con
  \begin{itemize}
  \item La imagen de tu modelo ER  con JDeveloper
  \item La imagen de tu modelo relacional con JDeveloper
  \item \texttt{\textbf{1.creacion.sql}}: El script de creación de tablas
  \item \texttt{\textbf{2.insercion.sql}}: El script de inserción de datos en las tablas
  \item \texttt{\textbf{3.modificacion.sql}}: El script de modificación de las tablas
  \item \texttt{\textbf{4.insercion.sql}}: El script de inserción de datos en las tablas modificadas
  \end{itemize}
\item Sube el documento a \enlace{http://aulavirtual2.educa.madrid.org/mod/assignment/view.php?id=1116703}{la tarea correspondiente en el aula virtual}
\item Presta atención al plazo de entrega (con fecha y hora).
  
\end{itemize}
\end{document}




