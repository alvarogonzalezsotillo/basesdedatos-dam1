\input{../common/plantilla-ejercicio.tex}
\usepackage{eurosym}
\usepackage{needspace}




\renewcommand{\hmwkTitle}{Práctica DBLinks}
\renewcommand{\hmwkClass}{Bases de datos}

\usepackage{blindtext}

\begin{document}

% \maketitle

% ----------------------------------------------------------------------------------------
%	TABLE OF CONTENTS
% ----------------------------------------------------------------------------------------

% \setcounter{tocdepth}{1} % Uncomment this line if you don't want subsections listed in the ToC

\primerapagina

\setlength{\parindent}{1em}
\setlength{\parskip}{1em}

\section{Objetivo de la práctica}
En esta práctica el alumno utilizará la funcionalidad \textit{dblink} de Oracle para implementar una base de datos distribuida.



\section{Descripción del problema}
Cada alumno representará una entidad bancaria. Cada entidad tendrá clientes, que pueden tener más de una cuenta.
Cada cuenta almacena movimientos. Un movimiento incluye el importe, una descripción y una marca de tiempo.


\begin{homeworkProblem}[: Vistas necesarias]
  No son importantes las tablas utlizadas para almacenar los datos, pero deben crearse las vistas del listado \ref{lst:vistas}, con los tipos de datos definidos en la tabla \ref{tbl:tipos}.



\begin{listadosql}{Creación de las vistas}{lst:vistas}
create or replace view v_cuentas(idcuenta) as
select ...;
create or replace view v_movimientos(idcuenta,marcadetiempo,euros,descripcion) as
select ...;
create or replace view v_clientes(idcliente,nombre) as
select ...;
\end{listadosql}


  \begin{figure}[h]
    \begin{tabular}{|c|c|p{10cm}|}
      \hline
      Campo & tipo & \\
      \hline
      \texttt{idcuenta}  & \texttt{VARCHAR(20)} & El identificador de cuenta. Los 4 primeros  indicarán el banco. Seguirá la cadena \texttt{0001DC}. Los últimos 10 seguirán una  secuencia. \\
      \texttt{idcliente} & \texttt{NUMBER(10)} & El identificador de cliente  \\
      \texttt{euros} & \texttt{NUMBER(15,2)} & Las cantidades en euros  \\
      \texttt{marcadetiempo} & \texttt{TIMESTAMP} & Las marcas de tiempo  \\
      \texttt{nombre} & \texttt{VARCHAR(255)} & Nombre del cliente \\
      \texttt{descripcion} & \texttt{VARCHAR(1024)} & Descripción del movimiento \\
      \hline
    \end{tabular}
    \caption{Tipos de datos de los campos}
    \label{tbl:tipos}
  \end{figure}

  
  
\end{homeworkProblem}


\newenvironment{Indentado}
{\hfill\minipage{\dimexpr\textwidth-2cm}}
{\endminipage}


\begin{homeworkProblem}[: Funciones internas del banco]

  \begin{itemize}
  \item \texttt{saldo\_cuenta(idcuenta)}

    \begin{Indentado}
      El saldo de una cuenta es la suma de los importes de sus movimientos. Una cuenta sin movimientos tiene saldo \texttt{0}. Si la cuenta no existe, se lanzará un error con el texto \texttt{CUENTANOEXISTE} 
    \end{Indentado}
    
  \item \texttt{saldo\_cliente(idcliente)}

    \begin{Indentado}
      El saldo de un cliente es la suma del saldo de todas las cuentas. Un cliente sin cuentas tiene saldo \texttt{0}. Si el cliente no existe, se lanzará el error \texttt{CLIENTENOEXISTE}
    \end{Indentado}


  \end{itemize}

\begin{listadosql}{Funciones internas del banco}{lst:funciones}
create or replace function SALDO_CUENTA( idcuenta IN varchar )
return number as begin
  ...
end; 

create or replace function SALDO_CLIENTE( idcliente IN number )
return number as begin
  ...
  RAISE_APPLICATION_ERROR(-20002, 'CLIENTENOEXISTE');
  ...
end; 

\end{listadosql}


\end{homeworkProblem}


\begin{homeworkProblem}[: Procedimientos internos del banco]


  \begin{itemize}
  \item \texttt{nuevo\_movimiento(idcuenta,euros,descripcion)}

    
    \begin{Indentado}
      Añade un movimiento a una cuenta, con la hora actual del sistema. Si la cuenta no existe, se lanzará el error  \texttt{CUENTANOEXISTE}.
    \end{Indentado}
    
  \item \texttt{crear\_cuenta(idcliente,idcuenta)}
    
    \begin{Indentado}
      Crea una cuenta cuyo titular es el cliente especificado. Si el cliente no existe se lanzará el error \texttt{CLIENTENOEXISTE}.  El identificador de cuenta será un \texttt{VARCHAR(20)}. Los 4 primeros indicarán el banco. Seguirá la cadena \texttt{0001DC}. Los últimos 10 seguirán una  secuencia.
    \end{Indentado}
    
    % \texttt{crear\_cuenta2(idcliente,idcuenta)}
    % Si \texttt{idcuenta} es \texttt{NULL}, crea una cuenta cuyo titular es el cliente especificado. Si el cliente no existe se lanzará el error \texttt{CLIENTENOEXISTE}. Si se especifica una cuenta, se añadirá el cliente a los titulares de esa cuenta. El identificador de cuenta será un \texttt{VARCHAR(20)}. Los 4 primeros caracteres indicarán el banco. Seguirá la cadena \texttt{0001DC}. Los últimos 10 seguirán una secuencia. \\


    

    
  \item \texttt{crear\_cliente(nombre,idcliente)}
    
    \begin{Indentado}
      Crea un nuevo cliente, devolviendo su identificador. \\
    \end{Indentado}
  \end{itemize}


  \begin{listadosql}{Procedimientos internos del banco}{lst:funciones}
create or replace procedure NUEVO_MOVIMIENTO(
  idcuenta IN varchar, 
  euros IN number,
  descripcion IN varchar)
as begin
  ...
  RAISE_APPLICATION_ERROR(-20002, 'CUENTANOEXISTE');
  ...
end; 

create or replace procedure CREAR_CUENTA(
  idcliente IN number
  idcuenta OUT varchar)
as begin
  ...
end; 

create or replace procedure CREAR_CLIENTE(
  nombre IN varchar, 
  idcliente OUT  number)
as begin
  ...
end; 


\end{listadosql}


\end{homeworkProblem}

\begin{homeworkProblem}[: Conexión con otros bancos]
  
  \begin{enumerate}
  \item Apunta el nombre de tu servidor, el usuario y la contraseña que usarán los demás bancos para conectarse tu base de datos en la hoja compartida 
    \begin{itemize}
    \item \enlace{https://docs.google.com/spreadsheets/d/1Hpo3YKgWtDsmaOB1p5PX5VdGO-c3HxdzpoBsCNI7cl4/edit}{https://docs.google.com/spreadsheets/d/ 1Hpo3YKgWtDsmaOB1p5PX5VdGO-c3HxdzpoBsCNI7cl4 /edit}.
    \item Recuerda que ese usuario debe tener privilegios para utilizar las vistas, funciones y procedimientos del ejercicio.
    \end{itemize}
  
  \item Crea los \textit{dblink} necesarios para conectarte a los demás bancos.

    
\end{enumerate}
\end{homeworkProblem}

\begin{homeworkProblem}[: Interacción entre bancos]
Cada banco podrá hacer transferencias entre sus cuentas. También podrá transferir dinero desde una de sus cuentas a una cuenta de otro banco. El procedimiento \texttt{TRANSFERENCIA( idcuentaorigen, euros, descripcion, idcuentadestino)} realizará las siguientes acciones
\begin{itemize}
\item En la cuenta de origen se hará un movimiento negativo (que reste dinero de la cuenta), usando el método \texttt{NUEVO\_MOVIMIENTO}.
\item En la cuenta de destino se hará un movimiento positivo (que añada dinero a la cuenta), usando el método \texttt{NUEVO\_MOVIMIENTO} del banco de destino via \textit{dblink}, si es necesario.
\item A la descripción de los momimientos se añadirá la cadena ``\texttt{cuentaorigen:\textit{la cuenta origen} cuentadestino:\textit{la cuenta destino}}''.
\item Si la cuenta de destino no es del propio banco, se hará un cargo de una comisión en la cuenta de origen. Este cargo será de un 2\% de la transferencia, con un mínimo de 3\euro.
\item El movimiento de la comisión tendrá como descripción ``\texttt{comision por transferencia de \textit{la cuenta origen} a \textit{la cuenta destino}}''  
\end{itemize}

Se tendrán en cuenta los siguientes casos de error:
\begin{itemize}
\item El banco de destino no existe: \texttt{BANCONOEXISTE}
\item La cuenta de origen no es del propio banco: \texttt{CUENTAAJENA}
\item La cuenta origen o destino no existe: \texttt{CUENTANOEXISTE}
\item La cuenta origen no tiene suficiente saldo: \texttt{SALDOINSUFICIENTE}
\item La transferencia es de un importe de 0{\euro} o menor: \texttt{IMPORTEINCORRECTO}
\end{itemize}

  \begin{listadosql}{Procedimientos internos del banco}{lst:funciones}
create or replace procedure TRANSFERENCIA(
  idcuentaorigen IN varchar, 
  euros IN number,
  descripcion IN varchar,
  idcuentadestino IN varchar )
as begin
  ...
  RAISE_APPLICATION_ERROR(-20002, 'IMPORTEINCORRECTO');
  ...
end; 

\end{listadosql}


\end{homeworkProblem}


\section{Instrucciones de entrega}
La autoría del trabajo es individual. Se corregirá \textit{on-line}, ejecutando pruebas mediante conexiones de red. Los servidores Oracle deberán estar funcionando y conectados en el día que el profesor pase dichas pruebas.



\end{document}




